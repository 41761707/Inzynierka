\begin{abstract}
    The main goal of a given thesis is to implement an algorithm for automated planning called \textbf{”GRAPHPLAN”}, which is based on formal language
    named \textbf{STRIPS}, that uses states (with a focus on two special states: initial and goal states) and actions which are used to make changes
    in the described world. The core of GRAPHPLAN is the usage of a structure called a \textbf{planning graph} which greatly reduces
    the amount of search needed during creation of a plan from initial states to goal states.

    Presented thesis consists of a detailed description of algorithm, application examples, implementation which uses a graph to show the optimal plan
    constructed from defined actions and changes made after each action in a given world, provides optional extensions, which depending on
    circumstances can have an impact on efficiency of an algorithm
    and tests, whose main goal is to show benefits and potential drawbacks of the usage of a given methodology of planning.
    
    Additionally, an integral part of this thesis is a concept called \textbf{constraint programming}, which heavily impacts on performance of a 
    described algorithm. This paper provides an exhaustive description with formal definitions and examples of a given way of programming and shows how 
    constraint programming is integrated into a planning graph.
\end{abstract}
    