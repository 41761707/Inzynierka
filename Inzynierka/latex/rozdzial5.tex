\chapter{Instalacja i wdrożenie}
\thispagestyle{chapterBeginStyle}

\textbf{UWAGA:} Poniższy opis przedstawia sposób instalacji odpowiednich pakietów dla komputerów korzystających z systemu operacyjnego 
\textbf{Linux}, a dokładniej- dystrybucji \textbf{Ubuntu}. Użytkownik chcąc zainstalować aplikację wraz z jej komponentami na 
komputerze z innym systemem operacjnym zobowiązany jest do samodzielnego zapoznania się ze wszystkimi komendami bądź mechanizami 
umożliwającymi instalację wskazanych pakietów.

\section{Instalacja pakietu SWI-Prolog}
\label{SWI-PROLOGRozdzial}
    Korzystając z dystrybucji Linuxa o nazie Ubuntu, wystarczającą czynnością do poprawnej instalacji pakietu SWI-Prolog jest uruchomienie 
    następującej komendy z poziomu linii komend wraz z prawami administratora 
    \begin{listing}[H]
        \begin{verbatim}
            sudo apt install swi-prolog-core
        \end{verbatim}
        \caption{Instalacja pakietu SWI-PROLOG z poziomu linii komend}
    \end{listing}
    W momencie, w którym komputer zakończy pobieranie oraz instalację pakietu wprowadzenie komendy \textit{swipl} powinno spodoować uruchomienie 
    interaktywnego interpretera języka PROLOG. Jeśli powyższa czynność zakończyła się sukcesem, komputer jest gotowy do uruchomienia kodu źródłowego 
    algorytmu oraz rozpoczęcia pracy nad kreowanie odpowiednich planów
\section{Instalacja języka Python}
    Większość dystrybucji Linuxa posiada wbudowany w sobie język programowania python. Z reguły można to zweryfikować poprzez wpisanie komendy 
    \begin{listing}[H]
        \begin{verbatim}
            python3 --version
        \end{verbatim}
        \caption{Komenda sprawdzająca wersję zainstalowanego języka python}
    \end{listing}
    Należy zauważyć, iż wszystkie komponenty zostały napisane dla wersji języka python 3.x. Użytkownik korzystając ze starszych wersji 
    może spotkać się z anomaliami negatywnie wpływającymi na funkcjonowanie aplikacji, dlatego zaleca się korzystanie ze wskazanej powyżej wersji.
    Do poprawnego uruchomienia aplikacji wymagane są następujaće biblioteki 
    \begin{itemize}
        \item Tkinter
        \item graphviz 
        \item PIL
        \item pyswip
    \end{itemize}
    Poniżej znajdują się odpowiednie komendy, który użycie z poziomu linii komend zagwarantuje poprawne uruchomienie aplikacji:
    \begin{listing}[H]
        \begin{verbatim}
            sudo apt install python3-tk
            pip3 install graphviz
            pip3 install Pillow
            pip3 install pyswip
        \end{verbatim}
        \caption{Instalacja odpowiednich bibliotek dla języka python}
    \end{listing}
    Przed uruchomienie powyższy komend użytkownik winien posiadać zainstalowany pakiet \textit{pip}. Jeśli pobieranie powyższych bibliotek zakończy 
    się błędem, należy uprzednio wykonać następującą komendę: texttt{sudo apt install python3-pip}
    Całą aplikację również uruchamia się z poziomu linii komend. Bo pobraniu odpowiednich plików, w folderze \texttt{sources} znajduje się plik
    \texttt{gui.py}, który zawiera kod rozruchowy interfejsu użytkownika. Będąc we wskazany katalogu należy uruchomić terminal i wprowadzić następującą
    komendę:
    \begin{listing}[H]
        \begin{verbatim}
            python3 gui.py
        \end{verbatim}
        \caption{Uruchomienie interfejsu użytkownika}
    \end{listing}
    Jeśli operacja zakończy się sukcese, użytkownik powinien ujrzeć okienko identyczne do tego, które zostało przedstawione w ramach sekcji \ref{GUIRozdzial}

\section{Dokumentacja}
