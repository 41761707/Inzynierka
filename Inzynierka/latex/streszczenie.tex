\begin{streszczenie}
Obiektem badań poniższej pracy jest metodologia planowania
o nazwie \textbf{"GRAPHPLAN"}, której esencją jest wykorzystanie
struktury zwanej \textbf{grafem planującym} w trakcie ustalania optymalnego planu transformacji
stanu początkowego w stan końcowy w ustalonej przestrzeni przy wykorzystaniu wcześniej zdefiniowanych
operatorów. \\
Dodatkowym aspektem pracy jest użycie programowania ograniczeń w celu zwiększenia wydajności jak i zmniejszenia
przestrzeni poszukiwań przez algorytm w trakcie generowania planu. \\
Ów praca składa się z formalnego opisu przytoczonego algorytmu, przedstawienia przykładów zastosowania,
implementacji, której wynikiem jest graf, przedstawiający optymalny plan wykonywanych operacji oraz zmian, jakie dzięki nim zachodzą w świecie,
omówienie opcjonalnych rozszerzeń, które w zależności od sytuacji mogą wpłynać na efektywność algorytmu
oraz przeprowadzonych testów, których zadaniem jest wskazanie mocnych, jak i słabych stron przedmiotu badań. \\
Grafy są generowane przy pomocy programu, który opiera się o moduł \textbf{graphviz}, dostępny w języku programowania \textbf{python}.

\end{streszczenie}
