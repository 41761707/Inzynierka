\begin{streszczenie}
    Obiektem badań poniższej pracy jest metodologia planowania
    o nazwie \textbf{"GRAPHPLAN"}, bazująca na formalnym języku \textbf{STRIPS}, skonstruowanym z myślą o przedstawianiu 
    rozpatrywanych światów w formie stanów (początkowego jak i końcowego), 
    oraz akcji między nimi, które wpływają na jego zmiany.
    Esencją GRAPHPLAN'u jest wykorzystanie
    struktury zwanej \textbf{grafem planującym} w trakcie ustalania optymalnego planu transformacji
    stanu początkowego w stan końcowy w ustalonej przestrzeni przy wykorzystaniu wcześniej zdefiniowanych
    operatorów. 

    Praca ta składa się z formalnego opisu przytoczonego algorytmu, przedstawienia przykładów zastosowania,
    implementacji, której wynikiem jest graf przedstawiający optymalny plan wykonywanych operacji oraz zmian, jakie dzięki nim zachodzą w 
    przedstawionym świecie, omówienie opcjonalnych rozszerzeń, które w zależności od sytuacji mogą wpłynać na efektywność algorytmu
    oraz przeprowadzonych testów, których zadaniem jest wskazanie mocnych, jak i słabych stron przedmiotu badań. 

    Dodatkowym aspektem pracy jest przedstawienie koncepcji o nazwie \textbf{programowanie ograniczeń}, 
    która została wykorzystana w celu usprawnienia osiągnięć czasowych algorytmu. 
    Wprowadzono formalne definicje programowania ograniczeń wraz z jego własnościami oraz przedstawiono w jaki sposób rzeczone podejście do 
    programowania zostało zawarte w implementacji badanego algorytmu.

\end{streszczenie}
