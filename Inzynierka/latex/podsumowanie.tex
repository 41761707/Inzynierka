\chapter*{Podsumowanie}
\addcontentsline{toc}{chapter}{Podsumowanie}
\thispagestyle{chapterBeginStyle}

    Główny cel pracy, którym była implementacja algorytmu GRAPHPLAN wraz z programowaniem ograniczeń, został osiagnięty. Przy pomocy 
    nowego podejścia do programowania udało się ułatwić proces implementacji jak i przyśpieszyć czas tworzenia planu przez algorytm. Użytkownik 
    może skorzystać z wbudowanych światów w ramach interfejsu graficznego, bądź uruchomić program z poziomu linii komend przy okazji 
    definiując własne środowisko pracy. Na żądanie użytkownika generowane są odpowiednie grafy reprezentujące plan, jak i schemat 
    transformacji świata z wyszczególnieniem sytuacji, w jakiej znajduje się każdy stan początkowy na dowolnym etapie realizacji planu.
    W ramach sprawdzenia funkcjonalności wykonano testy na prostych, lecz obrazowych przykładach. Na ich podstawie można wywnioskować, iż GRAPHPLAN,
    zgodnie ze swoim założeniem generuje plany, które są optymalne, czyli składają się jak najmniejszej liczby kroków. Wykonując testy na 
    popularnej \textit{przesuwance} można było zauważyć, iż rozszerzanie grafu planującego znacznie wpływa na długość wykonywania algorytmu. 
    Wprowadzenie świata z mniejszą liczbą stanów, w postaci ósemki, dokonało lekkiej poprawy sytuacji. 

    Algorytm posiada szereg możliwości pozwalających na rozszerzenie jego funkcjonalności oraz zwiększenie efektywności generowania planów. W trakcie 
    implementacji można zastosować mechanizm \textit{podwójnego szukania}, który miałby za zadanie układać plan symultanicznie korzystająć z mechanizmu 
    cofania oraz, na wzór ludzki, dynamicznie zmieniając świat próbując w ten sposób uzyskać wyznaczony cel. Dodatkowo wartym uwagi jest fakt, iż
    w niektórych sytuacjach można zrezygnować z gwarancji najkrótszego możliwego rozwiązania na rzecz szybszego generowania algorytmu. Ponadto 
    algorytm w swojej istocie bazuje głównie na relacji wzajemnego wykluczania. Istnieje możliwość, iż w zdefiniowanych światach istnieją inne informacje,
    które prowadziłyby do szybszego wyszukiwania odpowiedniego planu.

    Dalsze możliwe kierunki rozwoju oprogramowania to między innymi generowanie większej liczby grafów o różnych własnościach, które
    w dokładniejszy sposób prezentowałyby użytkownik esencję algorytmu, rozbudowa warty graficznej algorytmu pod kątem estetycznym jak i 
    pod kątem liczby światów w jakich użytkownik może generować plany.

