\chapter{Zawartość płyty CD}
\thispagestyle{chapterBeginStyle}
\label{plytaCD}

W załączonej płycie CD znajdują się wszystkie wymagane kody źródłowe. Poniżej przedstawiono schemat przedstawiający ułożenie katalogów
oraz plików na dołączonym nośniku danych:
\dirtree{%
.1 .
.2 sources.
.3 \hyperref[sources-graphplan.pl]{'graphplan.pl'}.
.3 \hyperref[sources-worlds.pl]{'worlds.pl'}.
.3 \hyperref[sources-gui.py]{'gui.py'}.
.3 \hyperref[sources-parsePlanFULL.py]{'parsePlanFULL.py'}.
.3 \hyperref[sources-parsePlanSIMPLIFIED.py]{'parsePlanSIMPLIFIED.py'}.
.3 graphs.
.4 \hyperref[graphs-FULLGRAPHPLAN.gv.png]{'FULLGRAPHPLAN.gv.png'}.
.4 \hyperref[graphs-SIMPLEGRAPHPLAN.gv.png]{'SIMPLE\_GRAPHPLAN.gv.png'}.
.3 outputs.
.4 \hyperref[outputs-parsePlanSIMPLIFIED.py]{'output.txt'}.
.3 html.
.4 \hyperref[html-index.html]{'index.html'}.
.3 tests.
.4 \hyperref[tests-8tests.txt]{'8tests.txt'}.
.4 \hyperref[tests-8testsmodified.txt]{'8testsmodified.txt'}.
.4 \hyperref[tests-15tests.txt]{'15tests.txt'}.
.4 \hyperref[tests-15testsmodified.txt]{'15testsmodified.txt'}.
.4 \hyperref[tests-cargotests.txt]{'cargotests.txt'}.
.2 praca.pdf.
}

