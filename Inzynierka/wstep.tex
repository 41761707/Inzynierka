%Korekta ALD - nienumerowany wstęp
%\chapter{Wstęp}
\addcontentsline{toc}{chapter}{Wstęp}
\chapter*{Wstęp}

\thispagestyle{chapterBeginStyle}

Praca dyplomowa inżynierska jest dokumentem opisującym zrealizowany system techniczny. Praca powinna być napisana poprawnym językiem odzwierciedlającym aspekty techniczne (informatyczne) omawianego zagadnienia. Praca powinna być napisana w trybie bezosobowym (w szczególności należy unikać trybu pierwszej osoby liczby pojedynczej i mnogiej). Zdania opisujące konstrukcję systemu informatycznego powinny być tworzone w stronie biernej. W poniższym dokumencie przykłady sformułowań oznaczono kolorem niebieskim. W opisie elementów systemu, komponentów, elementów kodów źródłowych, nazw plików, wejść i wyjść konsoli należy stosować czcionkę stałej szerokości, np: {\color{lgray}zmienna \verb|wynik| przyjmuje wartość zwracaną przez funkcję \verb|dodaj(a,b)|, dla argumentów \verb|a| oraz \verb|b| przekazywanych \ldots}.

Układ poniższego dokumentu przedstawia wymaganą strukturę pracy, z rozdziałami zawierającymi analizę zagadnienia, opis projektu systemu oraz implementację (dobór podrozdziałów jest przykładowy i należy go dostosować do własnej tematyki pracy). 
  
Wstęp pracy (nie numerowany) powinien składać się z czterech części (które nie są wydzielane jako osobne podrozdziały): zakresu pracy, celu, analizy i porównania istniejących rozwiązań oraz przeglądu literatury, oraz opisu zawartości pracy.

Każdy rozdział powinien rozpoczynać się od akapitu wprowadzającego, w którym zostaje w skrócie omówiona zawartość tego rozdziału.

{\color{dgray}
Praca swoim zakresem obejmuje wielowarstwowe rozproszone systemy informatyczne typu ,,B2B'' wspierające wymianę danych pomiędzy przedsiębiorstwami. Systemy tego typu, konstruowane dla dużych korporacji, charakteryzują się złożoną strukturą poziomą i pionową, w której dokumenty \ldots

Celem pracy jest zaprojektowanie i oprogramowanie aplikacji o następujących założeniach funkcjonalnych:
\begin{itemize}
    \item wspieranie zarządzania obiegiem dokumentów wewnątrz korporacji z uwzględnieniem \ldots,
	\item wspieranie zarządzania zasobami ludzkimi z uwzględnieniem modułów kadrowych oraz \ldots,
	\item gotowość do uzyskania certyfikatu ISO \ldots,
	\item \ldots.
\end{itemize}

Istnieje szereg aplikacji o zbliżonej funkcjonalności: \ldots, przy czym \ldots.

Praca składa się z czterech rozdziałów.
W rozdziale pierwszym omówiono strukturę przedsiębiorstwa \ldots, scharakteryzowano grupy użytkowników oraz przedstawiono procedury związane z obiegiem dokumentów. Szczegółowo opisano mechanizmy \ldots. Przedstawiono uwarunkowania prawne udostępniania informacji \ldots, w ramach \ldots.

W rozdziale drugim przedstawiono szczegółowy projekt systemy w notacji UML. Wykorzystano diagramy \ldots.
Opisano w pseudokodzie i omówiono algorytmy generowania \ldots.

W rozdziale trzecim opisano technologie implementacji projektu: wybrany język programowania, biblioteki, system zarządzania bazą danych, itp.  Przedstawiono dokumentację techniczną kodów źródłowych interfejsów poszczególnych modułów systemu. Przedstawiono sygnatury metod publicznych oraz \ldots.

W rozdziale czwartym przedstawiono sposób instalacji i wdrożenia systemu w środowisku docelowym.

Końcowy rozdział stanowi podsumowanie uzyskanych wyników.
}

