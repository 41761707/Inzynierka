%Korekta ALD - nienumerowany wstęp
%\chapter{Wstęp}
\addcontentsline{toc}{chapter}{Wstęp}
\chapter*{Wstęp}

\thispagestyle{chapterBeginStyle}

Celem pracy jest zaimplementowanie algorytmu do planowania akcji o nazwie \textbf{GRAPHPLAN}, który po raz pierwszy został sformalizowany i opisany w pracy pod tytułem
\textbf{"Fast Planning Through Planning Graph Analysis"}\cite{GRAPHPLAN} przez Panów Avrim L. Blum'a i Merrick L. Furst'a.
\\
Praca składa się z pięciu rodziałów.
\\

	W rozdziale pierwszym poruszono aspekty historyczne odnośnie planowania akcji przy użyciu komputerów oraz jaką rolę pełni w niej GRAPHPLAN,
dokonano teoretycznego porównania algorytmu względem nowoczesnych metod z przytoczonej dziedziny informatyki. Ponadto przedstawiono dlaczego
w naturalny sposób do planowania akcji przedsięwziono grafy.
\\

	W rodziale drugim poddano dogłębnej analzie implementowany algorytm- dokładnie opisano jego strukturę, warstwy, z których się składa oraz własności,
które wyróżniają go na tle innych rozwiązań problemów związanych z planowaniem. W celu łatwiejszego przyswojenia mechanizmów stojących
za GRAPHPLAN'em w trakcie opisu wprowadzono liczne proste przykłady wraz z grafikami wygenerowanymi przy pomocy narzędzi stworzonych
na potrzeby ów pracy.
\\

	W rozdziale trzecim rozwinięto pojęcie programowania ograniczeń, wprowadzając formalną definicję, podstawowe słownictwo niezbędne do 
	zrozumienia idei stojącej za tym sposobem programowania, omówiono benefity płynące z wykorzystania tego podejścia oraz przedstawiono
	obrazowo schemat funkcjonowania na podstawie prostych przykładów.
\\

	Rodział czwarty skupia się na szczegółach implementacyjnych: wybranych jezykach programowania oraz technologiach wykorzystywanych również
w warstwach graficznych programu. Dokonano szczegółowego opisu interfejsu użytkownika oraz jego możliwości, połączeń między komponentami oraz 
ważniejszych funkcji stanowiących trzon pracy.
\\

	Rodział piąty przedstawia sposób instalacji oraz instrukcję obsługi programu, dodatkowo zawiera instrukcję odnośnie instalowania wszystkich
niezbędnych komponentów wykorzystywanych w pracy, w których skład wchodzą interpretery jak i kompilatory używanych języków programowania oraz 
wszystkie biblioteki i moduły.
\\

	Rodział szósty przedstawia przeprowadzone testy, które badają możliwości algorytmu w wcześniej spreparowanych środowiskach. W tej części została
przeprowadzona analiza wydajnościowa algorytmu, weryfikacja wygenerowanych planów pod względem poprawności oraz porównanie otrzymanych wyników
z innymi powszechnie wykorzystywanymi metodami planowania. Każdy z testów zawiera w sobie wniosek, w którym odbywa się zbiorcza
ocena wszystkich wyżej wymienionych aspektów.
\\

	Końcowy rozdział stanowi zbiorcze podsumowanie pracy z komentarzem odnośnie potencjalnych rejonów, w których algorytm mógłby znaleźć swoje
zastosowanie. 

