\chapter{Programowanie ograniczeń}
\thispagestyle{chapterBeginStyle}

\section{Wprowadzenie}
    Programowanie ograniczeń (inna nazwa: technologia więzów) jest narzędziem wykorzystywanym do rozwiązywania 
    problemów z dziedzin kombinatoryki, sztucznej inteligencji, czy planowania jak i harmonogramowania zadań.
    W składzie tego podejścia do programowania często wyróżnia się dwa elementy: ograniczenia (zwane również
    stałymi) oraz problem rozwiązywania ograniczeń (ang. Constraint Satisfaction Problem, CSP)
    Poniżej dokonano formalnego zdefiniowania powyższych komponentów:
    \begin{definition}
        \label{ConstraintProblem}
        \textbf{Problem rozwiązywania ograniczeń, CSP} jest następującą trójką:
        \begin{equation}
            CSP = (V,D,C)
        \end{equation}
        gdzie:

        $V = \{x_{1},x_{2},...,x_{n}\}$ oznacza zbiór zmiennych wykorzystywanych do opisu problemu

        $D = \{d_{1},d_{2},...,d_{n}\}$ oznacza zbiór dziedzin wyżej wspominanych zmiennych. W ramach 
        rozważeń zawartych w rzeczonej pracy rozpatrywane będą takie $d_{i}$, które są zbiorami zawierającymi 
        skończoną liczbę potencjalnych wartości zmiennej $x_{i}$.

        $C = \{c_{1},c_{2},...,c_{m}\}$ oznacza zbiór ograniczeń
    \end{definition}
    \begin{definition}
        \label{Constraint}
        \textbf{Ograniczenami} (inne nazwy: stałe, więzy) nazywamy zmienne oraz zależności między nimi, które muszą zostać spełnione 
        w ramach rozwiązywania problemu ograniczeń. Określamy je przy pomocy pary: 
        \begin{equation}
            C = (S,R)
        \end{equation}
        gdzie:

        S jest krotką wszystkich zmiennych wchodzących w skład relacji 

        R jest relacją, która definiuje jakie wartości mogą przyjąc zmienne, które w niej uczestniczą.
    \end{definition}
    Relacje często przedstawia się przy pomocy zbioru zawierającego krotki, które składają się ze wszystkich 
    przyporządkowań wartości do odpowiednich zmiennych.
    Przy następujących definicjach oczekiwanym celem będzie utworzenie mechanizmu rozwiązującego zadany problem. Jego wynikiem 
    będzie zbiór wszystkich zmiennych, oraz krotek, które będą zawierały odpowiednie wartości przyporządkowane dla zmiennych.
    \begin{definition}
        \label{Krotka}
        \textbf{Krotka} (ang. tuple) - struktura danych, która w systemach informatycznych odzwierciedla uporządkowany ciąg wartości
    \end{definition}
    Dodatkowo wprowadza się termin \textbf{arności} ograniczenia:
    \begin{definition}
        \label{Krotka}
        \textbf{Arnością ograniczenia} (ang. arity)  związana jest z liczbą unikalnych zmiennych, która w nią wchodzi.
    \end{definition}
    CSP 
    Najpopularniejszymi typami ograniczeń są:
    \begin{enumerate}
        \item Ograniczenia o arności 1, zwane ograniczeniami \textbf{unarnymi} (w tym przypadku z reguły będą one ściśle związane z dziedziną, raczej 
        nie będą występowały w kontekście ograniczeń)
        \item Ograniczenia o arności 2, zwane ograniczeniami \textbf{binarnymi}
        \item Ograniczenia o arności 3, zwane ograniczeniami \textbf{ternarnymi}
    \end{enumerate} 

    Tak jak ograniczenie może mieć swoją arność, tak dla CSP również zdefiniowano pojęcie arności w lekko zmodyfikowany sposób 
    \begin{definition}
        \textbf{Arność} CSP o wartości $i$ zawiera w sobie wszystkie typy ograniczeń od arności 1 aż do arności $i$
    \end{definition}

    Wedle powyższego binarne CSP zawiera w sobie jedynie ograniczenia unarne jak i binarne.
    \begin{example}
        \label{CP1}
        Niech będzie dane równanie $x+y=z$, gdzie $x,y,z \in \{0,1\}$. Łatwo zauważyć, iż zadane równanie jest automatycznie ograniczeniem 
        wpływającym na prezentowane zmienne. Przyporządkowując odpowiednie wartości do zbiorów z definicji \label{ConstraintProblem} otrzymano
        \begin{enumerate}
            \item $ V = \{x,y,z\} $
            \item $ D = \{d_{x},d_{y},d_{z}\}$, gdzie $d_{x},d_{y},d_{z} = \{0,1\}$
            \item $ C = \{x+y=z\}$
        \end{enumerate}
        \textbf{Arność} ograniczenia występujące w przedstawionym przykładzie wynosi 3, determinowane jest to liczbą zmiennych, która wchodzi w jej skład.
        Rozwiązaniem tego problemu będą następujące dopasowania:

        
        $((x,y,z),\{0,0,0\}, \{1,0,1\}, \{0,1,1\})$

    \end{example}
    Odnalezienie rozwiązań z przykładu \ref{CP1} było trywialne ze względu na małą liczbę zmiennych, wąskie dziedziny oraz tylko jedno wprowadzone ograniczenie.
    Podobnie jak z planowaniem, wprowadzenie dodatkowych zmiennych, powiększanie dziedzin oraz zbioru ograniczeń znacznie wpływa na skomplikowanie 
    odnajdowania rozwiązania.



\section{Pojęcie ustalenia i spójności}
    \label{SpójnośćRodział}
    W ostatecznej formie problem ograniczeń wyszukuje rozwiązanie przy pomocy ustalenia wartości zmiennych, jednakże naistotniejsza 
    jest droga, jaką pokonuje, aby ów ustalenia uzyskać. W tej sekcji należy wprowadzić kilka dodatkowych definicji:
    \begin{definition}
        Ustalenie zmiennych jest \textbf{spójne}, gdy nie jest w konflikcie z żadnym z ograniczeń.
    \end{definition}
    Ustalenie spójne również w nomenlakturze programowania ograniczeń nazywane jest ustaleniem \textbf{legalnym}.
    Nadanie zmiennym X oraz Y wartość 1 w przykładzie \ref{CP1} prowadzi do konfliktu, gdyż nie istnieje wartość 2 w zbiorze $D_{z}$.
    \begin{definition}
        \textbf{Kompletnym ustaleniem} nazywamy takie ustalenie, w którym wszystkie zmienne posiadają ustaloną wartość.
    \end{definition}
    Łatwo zauważyć, iż kompletne ustawienie jest jednocześnie rozwiązaniem problemu ograniczeń. Z tego płynie następujący wniosek:
    \begin{corollary}
        Każde rozwiązanie problemu ograniczeń jest spójne.
    \end{corollary}

    Dodatkowo w źródłach \ref{AI} definiuje się częściowe ustalenie oraz częściowe rozwiązanie. Zgodnie z nazewnictwem 
    częściowe ustalenie związane jest z sytuacją, gdy jeszcze nie wszystkie zmienne mają dokładnie określone wartości, natomiast 
    częściowe rozwiązanie w praktyce identyfikuje się jako spójne częściowe ustalenie.

    Po wprowadzeniu powyższych definicji należy rozpocząć rozważań na temat tego, w jaki sposób 
    problem ograniczeń może zostać rozwiązany. Pierwszym z podejść może być ustalenie wartości dla zmiennych poprzez 
    analizę ograniczeń, jakie między nimi występują. Ten proces nazywany jest \textbf{propagacją ograniczenia}. Dzięki 
    podstawowej analizie ograniczeń algorytm może wyeliminować wartości nadmiarowe znajdujące się w zadanych dziedzinach, co znacząco wpłynie 
    na przyszłościowe osiągi pod kątem czasowym. Często propagacja ograniczeń jest wykonywana jako \textit{preprocessing step}, czyli 
    jako krok, który zostanie wykonany jeszcze przed rozpoczęciem prawdziwej pracy nad problem. 
    
    Kluczem do uzyskania poprawnego efektu propagacji ograniczeń jest skorzystanie z pojęcia zdefiniowanego jako \textbf{lokalna spójność}.
    Istnieją różne typy lokalnej spójności:

\section{Wyszukiwanie rozwiązań}


\section{Obrazowe przykłady}

//SEND + MORE = MONEY, N HETMANÓW, czy problem plecakowy?
//Wszystkie trzy będą pewnie zbędne

\section{Wykorzystanie w algorytmie}

    Podczas graficznego prezentowania przykładów programowania ograniczeń często wykorzystywaną strukturą był graf, chociażby w sekcji omawiającej 
    pojęcie lokalnych spójności (\ref{SpójnośćRodział}). Ze względu na ów powiązanie między programowaniem ograniczeń a GRAPHPLAN'em do podstawowego opisu GRAPHPLANU z rodziału 2 
    \ref{GRAPHPLANRozdzial} dodano funkcjonalności opisane w powyższych rozdziałach. 

    Każdy ze stanów oraz akcji zawiera w sobie dodatkowy \textbf{indykator stanu}. Jest to liczba z zakresu $\{0,1\}$, o której 
    należy mysleć bardziej w kontekście wartości bool'owskich $\{prawda,fałsz\}$. Przy pomocy indykatorów program ustala, które stany, bądź akcje 
    są w danej warstwie prawdziwe, czyli występują w świecie oraz takie, które w ów świecie w danym momencie nie występują, czyli są fałszywe. 
    

