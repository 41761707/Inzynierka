\chapter{Programowanie ograniczeń}
\thispagestyle{chapterBeginStyle}

\section{Wprowadzenie}
    Programowanie ograniczeń (inna nazwa: technologia więzów) jest narzędziem wykorzystywanym do rozwiązywania 
    problemów z dziedzin kombinatoryki, sztucznej inteligencji, czy planowania jak i harmonogramowania zadań.
    W składzie tego podejścia do programowania często wyróżnia się dwa elementy: ograniczenia (zwane również
    stałymi) oraz problem rozwiązywania ograniczeń (ang. Constraint Satisfaction Problem, CSP)
    Poniżej dokonano formalnego zdefiniowania powyższych komponentów:
    \begin{definition}
        \label{ConstraintProblem}
        \textbf{Problem rozwiązywania ograniczeń, CSP} jest następującą trójką:
        \begin{equation}
            CSP = (V,D,C)
        \end{equation}
        gdzie:

        $V = \{x_{1},x_{2},...,x_{n}\}$ oznacza zbiór zmiennych wykorzystywanych do opisu problemu

        $D = \{d_{1},d_{2},...,d_{n}\}$ oznacza zbiór dziedzin wyżej wspominanych zmiennych. W ramach 
        rozważeń zawartych w rzeczonej pracy rozpatrywane będą takie $d_{i}$, które są zbiorami zawierającymi 
        skończoną liczbę potencjalnych wartości zmiennej $x_{i}$.

        $C = \{c_{1},c_{2},...,c_{m}\}$ oznacza zbiór ograniczeń
    \end{definition}
    \begin{definition}
        \label{Constraint}
        \textbf{Ograniczenami} (inne nazwy: stałe, więzy) nazywamy zmienne oraz zależności między nimi, które muszą zostać spełnione 
        w ramach rozwiązywania problemu ograniczeń
        Przy następujących definicjach oczekiwanym celem będzie utworzenie mechanizmu rozwiązującego zadany problem. Jego wynikiem 
        będzie zbiór wszystkich zmiennych, oraz krotek, które będą zawierały odpowiednie wartości przyporządkowane dla zmiennych.
    \end{definition}
    \begin{definition}
        \label{Krotka}
        \textbf{Krotka} (ang. tuple) - struktura danych, która w systemach informatycznych odzwierciedla uporządkowany ciąg wartości
    \end{definition}
    Dodatkowo wprowadza się termin \textbf{arności} ograniczenia:
    \begin{definition}
        \label{Krotka}
        \textbf{Arnością} (ang. arity) ograniczenia związana jest z liczbą unikalnych zmiennych, która w nią wchodzi.
    \end{definition}

    Najpopularniejszymi typami ograniczeń są:
    \begin{enumerate}
        \item Ograniczenia o arności 1, zwane ograniczeniami \textbf{unarnymi} (w tym przypadku z reguły będą one ściśle związane z dziedziną, raczej 
        nie będą występowały w kontekście ograniczeń)
        \item Ograniczenia o arności 2, zwane ograniczeniami \textbf{binarnymi}
        \item Ograniczenia o arności 3, zwane ograniczeniami \textbf{ternarnymi}
    \end{enumerate} 

    \begin{example}
        \label{CP1}
        Niech będzie dane równanie $x+y=z$, gdzie $x,y,z \in \{0,1\}$. Łatwo zauważyć, iż zadane równanie jest automatycznie ograniczeniem 
        wpływającym na prezentowane zmienne. Przyporządkowując odpowiednie wartości do zbiorów z definicji \label{ConstraintProblem} otrzymano
        \begin{enumerate}
            \item $ V = \{x,y,z\} $
            \item $ D = \{d_{x},d_{y},d_{z}\}$, gdzie $d_{x},d_{y},d_{z} = \{0,1\}$
            \item $ C = \{x+y=z\}$
        \end{enumerate}
        \textbf{Arność} ograniczenia występujące w przedstawionym przykładzie wynosi 3, determinowane jest to liczbą zmiennych, która wchodzi w jej skład.
        Rozwiązaniem tego problemu będą następujące dopasowania:

        
        $((x,y,z),\{0,0,0\}, \{1,0,1\}, \{0,1,1\})$

    \end{example}
    Odnalezienie rozwiązań z przykładu \ref{CP1} było trywialne ze względu na małą liczbę zmiennych, wąskie dziedziny oraz tylko jedno wprowadzone ograniczenie.
    Podobnie jak z planowaniem, wprowadzenie dodatkowych zmiennych, powiększanie dziedzin oraz zbioru ograniczeń znacznie wpływa na skomplikowanie 
    odnajdowania rozwiązania.

\section{Wyszukiwanie rozwiązań}

\section{Obrazowe przykłady}

//SEND + MORE = MONEY, N HETMANÓW, czy problem plecakowy?
//Wszystkie trzy będą pewnie zbędne

