\begin{streszczenie}
Obiektem badań poniższej pracy jest metodologia planowania
o nazwie "Graphplan", której esencją jest wykorzystanie
własności grafów w trakcie ustalania optymalnego planu transformacji
stanu początkowego w stan kończowy w ustalonej przestrzeni przy wykorzystaniu wcześniej zdefiniowanych
operatorów. \\
Ów praca składa się z formalnego opisu przytoczonego algorytmu, przedstawienia przykładów zastosowania,
implementacji, której wynikiem jest graf, przedstawiający optymalny plan wykonywanych operacji,
omówienie opcjonalnych rozszerzeń, które w zależności od sytuacji mogą wpłynać na efektywność algorytmu
oraz przeprowadzonych testów, których zadaniem jest wskazanie mocnych, jak i słabych stron przedmiotu badań.

\end{streszczenie}
