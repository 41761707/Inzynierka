\chapter{Wprowadzenie}
\thispagestyle{chapterBeginStyle}
\label{rozdzial1}
\section{Planowanie}

    Codziennie ludzie używają słowa "plan" w różnych kontekstach: firma planuje rozkład palet na magazynie, trener planuje strategię na
    najbliższy mecz, student planuje jak rozwiązać zadanie na kolokwium, czy chociażby człowiek codziennie planuje listę zakupów. \\
    Mimo różnych kontekstów istnieją wyodrębnione wspólne cechy każdego z planów:
    \begin{definition}
    \label{StanyPoczatkowe}
        \textbf{Warunki początkowe} - stan świata przed zastosowaniem jakichkolwiek akcji. W dalszej części pracy również określane jako 
        \textbf{stany początkowe}.
    \end{definition}
    \begin{itemize}
        \item Każdy plan musi mieć jasno zadeklarowane warunki początkowe.
        Dzięki dokładnej wiedzy o świecie możliwym jest poprawne określenie akcji, przy pomocy których wprowadzane są modyfikacje
        obecnego stanu aż do otrzymania zadowalających rezultatów. Dla przykładu, firma musi wiedzieć ile oraz jakie palety 
        przybędą na magazyn zanim rozpocznie planowanie rozkładu dostawy na magazynie.
    \end{itemize}
    \begin{definition}
    \label{Akcje}
        \textbf{Akcja} - działanie zmieniające przedstawiony świat w ściśle określony sposób. W dalszej części pracy również określane jako
        \textbf{operator}.
    \end{definition}
    \begin{itemize}
        \item Akcje - akcje pozwalają na modyfikację przedstawionego świata. Każda z akcji składa się z podmiotu, na który działa oraz czynności,
        która jest względem wskazanego podmiotu wykonywana. Przykładem dobrze określonej akcji może być przeniesienie klocka z 
        jednego stolika na drugi- składa się ona z podmiotu w postaci klocka, oraz czynności w postaci przenoszenia, które możemy traktować
        jako ruch. Czynności mogą różnić się od siebie w kwestii skomplikowania, najważniejszym jest, aby były określone poprawnie, co
        sprowadza się do tego, aby były wykonalne w zdefiniowanym świecie..
    \end{itemize}
    \begin{definition}
    \label{Cel}
        \textbf{Cel} - Oczekiwany stan świata.
    \end{definition}
    \begin{itemize}
        \item Cele - kwintesencją każdego planu jest cel, który należy uzyskać. Zwyczajowo plany składają się z celów możliwych do
        osiągnięcia ze stanu początkowego przy pomocy zdefiniowany operacji, jednakże trzeba wziąc pod uwagę sytuację, w której 
        niemożliwym jest uzyskanie wskazanego celu, szczególnie próbująć automatyzować pojęcie planowania.
    \end{itemize} 
    Przy pomocy powyższych definicji możliwym jest sformalizowanie pojęcia stojącego za słowem \textbf{plan}. 
    \begin{definition}
    \label{Plan}
    \textbf{Plan}- lista akcji, której zastosowanie do stanu początkowego powoduje jego zmianę do stanu określonego w ramach cel. 
    \end{definition}


\section{Planowanie przy użyciu komputerów}
    
\section{Użycie grafów- motywacja}




